\documentclass[a4paper,12pt]{article}
\pdfoutput=1 
\usepackage{jheppub}
\usepackage[T1]{fontenc} % if needed

\title{\boldmath Constraints on non-Standard Model vector boson ineractions 
		in an effective field theory using differencial cross section of the production 
		of two jets associate with two leptonically decayed Z bosons at 
		$\sqrt{s} = 13$ TeV with the ATLAS detector}
\author{Dong Qichen}
\affiliation{The University of Manchester,\\316 Oxford Road, UK}
\emailAdd{dong.qichen@student.manchester.ac.uk}
\abstract{Abstract...}


\begin{document} 
	\maketitle
	\flushbottom

	\section{Introduction}
	\label{sec:intro}
		\par Since the discovery of last particle predicted by the Standard Model 
		of particle physics (SM)\cite{Gaillard_1999}, the Higgs boson, in 2012\cite{Aad_2012}, 
		the nature of its interactions with other particles has been carefully investigated in order 
		to seek any deviation from the prediction of the SM. The Vector Boson 
		Scattering (VBS) process\cite{rauch2016vectorboson}, a important pure 
		electroweak process in the Large Hardron Collidor (LHC)\cite{Evans_2008}, 
		enables us to probe the nature of Higgs 
		boson interactions with vector bosons and to constrain Beyond the Standard 
		Model (BSM) hypothesis altering the nature of the Higgs boson and/or vector bosons
		which may provide alternative Electroweak Symmetry Breaking (EWSB) 
		machenism\cite{dawson1999introduction} and additional source of CP 
		violation\cite{peccei1995cp}.
		
		\noindent\par In the LHC, VBS events are produced by a pair of vector bosons radiated 
		from the colliding partons scattered to another pair of vector bosons subsequently.
		At the ATLAS detector-level, the signiture of VBS events are two jets from the 
		initial partons associate with the decay products of two vector boson, we denote such 
		events $VVjj$. The evidence of electroweak production of $W^{\pm}W^{\pm}jj$
		($EW W^{\pm}W^{\pm}jj$) was first observed during the first run of LHC with the 
		ATLAS detector\cite{Aad_2014}, and the observation of this very process has been 
		comfirmed by the CMS\cite{Chatrchyan:2008aa} collaboration during the second 
		run of LHC in 2018\cite{Sirunyan_2018}, while the observation of $EW\ ZZjj$ channel 
		has been reported in 2019\cite{ATLAS-CONF-2019-033}. Inspite the endeavour 
		has been put on the VBS process, the model-independent measurement of the differencial 
		cross section have not been made so far.

		\par With a total of $139\ fb^{-1}$ proton proton collision data at 
		$\sqrt{s} = 13$ TeV acummulated in the ATLAS detector\cite{Aad:2008zzm} 
		during the second run of the LHC,
		a great opportunity to scrutinise the SM at much higher accuracy has been provided.
		In this report, the fiducial differencial cross section of $ZZjj$ process measured at 
		$\sqrt{s} = 13$ TeV using the whole run II data is reported and the results has 
		been used to constrain a BSM effective field theory\cite{Brivio_2017} which provides 
		alternative Higgs boson interactions, vector boson interactions and additional 
		source of CP violation in the electroweak sector.

	\section{Recap}
	\label{sec:recap}
		\emph{36/fb unfolded results for dpjj distributions}
	\section{Differential ZZjj cross section measurements}
	\label{sec:xsmea}
		\subsection{Theory}
		\label{ss:theory}
			\emph{i.e. contributing processes (EW ZZjj, strong ZZjj, what the backgounds processes are)}
		\subsection{Chosen Observables}
		\subsection{simulations}
		\subsection{Event selection}
		\subsection{Unfolding}
		\subsection{Results}
	\section{Reinterpretation in an effective field theory}
		\subsection{EFT theory}
		\subsection{Truth-level event generation}
		\subsection{Limit setting procedure}
		\subsection{Results}
	
	\section{Future works}
	\label{sec:future}
		\emph{(you miss many systematics in each part of the results, so give some thought about what you would do if given another 6months on this research).}
	\section{Conclusions}
	\label{sec:conc}

	\clearpage
	\bibliographystyle{JHEP}
	\bibliography{bibli.bib}
\end{document}
